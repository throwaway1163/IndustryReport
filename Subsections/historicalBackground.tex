The history of the software industry has been characterized by constant and often abrupt change that is driven by steady improvements in underlying technologies and the growth and maturation of the new sectors within the industry created by these developments. Early software was usually bundled with hardware as a single integrated product.\autocite[4]{buxmann2012software}
Oftentimes software was written from scratch for each new computer architecture, or in some instances, for each customer.\autocite{MainframesStillAround}
Computing was dominated by large and expensive mainframes like those sold and supported by IBM; in this model all the actual computing was carried out in central processors and the access points to control the computers were dumb terminals.\autocite[]{LargeParadigmShiftCloudComputing}
This model was necessarily only available to large business and institutions capable of purchasing and maintaining such an expensive piece of equipment.
The personal computer broke this model and pushed most of the industry towards the client/server architecture where much of the processing takes place directly on the user's terminal.\autocite[]{LargeParadigmShiftCloudComputing}

Mainframes still around\autocite{MainframesStillAround}

The first IBM PC was launched in 1981 and was priced at \$1,565, targeting the consumer market with promotional materials touting such user-friendly features as "easily understood operation manuals" and emphasizing that it was "possible to begin using the computer within hours."\autocite{PCBirthday}
Ironically, by focusing on carefully documenting each component of the PC and using mostly off-the-shelf parts, IBM inadvertently opened the doors to competing hardware manufacturers which went on to undercut the original IBM PC.\autocite{PCBirthday}
Ultimately, Microsoft and Intel came to be the greatest beneficiaries of the IBM PC and continue to rule the PC market to this day.\autocite{PCBirthday}

Larry Ellison started Software Development Laboratories, Inc. in 1977 with his partners Robert Miner and Edward Oates.\autocite[113]{finkle2012larry}
They changed the company name to Relational Software, Inc. and then to Oracle.\autocite[113]{finkle2012larry}
Oracle was one of the first companies to begin selling relational database management systems, two years before IBM introduced their own systems.\autocite[113]{finkle2012larry}
Oracle's profitability grew rapidly and by 1982 the company had 24 employees and \$2.5 million annual revenue.\autocite[113]{finkle2012larry}
Oracle went public in 1986 and Oracle reached revenues of \$55 million.\autocite[113]{finkle2012larry}
In 1990 Oracle began to post losses and saw its market capitalization fall by 80\%.\autocite[113]{finkle2012larry}
Ellison refocused the company, fired many of the senior managers and replaced them with more experienced hires while he focused more on product development.\autocite[113]{finkle2012larry}
The Oracle 7 Database Management System released in 1992 took the industry by storm as Oracle's profits shot up by 76\% within a single quarter.\autocite[114]{finkle2012larry}

The family of operating systems commonly referred to as Linux trace much of their roots back to the GNU project, started by Richard Stallman in 1983 with his essay, ``The GNU manifesto.''\autocite[]{GNUManifesto}
In it Stallman declared his intention to create a free operating system denouncing the current trend towards proprietary software development which he believed to be unethical.\autocite[]{GNUManifesto}
The system was to be modeled upon and compatible with the older UNIX, but built up entirely of freely licensed software(GNU is actually a recursive acronym sanding for GNU is Not Unix).\autocite[]{GNUManifesto}

In 1995 Netscape Navigator had more than 80\% market share. However, when Microsoft decided to enter the market it was able to leverage its deep pockets and control of the windows platform to invest in and distribute the Internet Explorer browser.\autocite[27]{buxmann2012software}
By 1998 Netscape's market share had collapsed falling to bellow 4\% and the navigator browser was open-sourced, creating the Mozilla project.\autocite[27]{buxmann2012software}
Without competition the Internet Explorer project stagnated while Mozilla continued to develop after being embraced by the open-source community and eventually reemerged as Firefox in 2004 with far more features than Internet Explorer and was able to claim significant market share.\autocite[27]{buxmann2012software}

From 1997 to 2000 there was a buying frenzy in the business software industry as the Y2K panic drove greater investment in enterprise software.\autocite[]{AftermathOfIntegratedStack}
This enthusiasitic purchasing diminshed somewhat as business struggled to integrate and make effective use of their investments.\autocite[]{AftermathOfIntegratedStack}
Enterprise software has had a substantial impact upon business over the past few years as these integration efforts have begun to pay off. Retail for example has seen historically low inventories and fewer seasonal hires, arguably as a result efficiencies gained through the effective use of enterprise resource management software. \autocite[]{AftermathOfIntegratedStack}
Companies have now shifted towards what are called integrated stacks, software packages that combine as many of the essentially components of the computer system's architecture as possible.
Brining together diverse tools into a single product that unifies hardware, middleware, virtualization layer, operating system, and enterprise applications.\autocite[]{AftermathOfIntegratedStack}

The rise of smart phones has seen the creation of App Stores turning most popular mobile devices into software distribution platforms.\autocite[]{TheAppEconomy}
These app stores have seen some remarkable successes as app makers producing everything from small games to enterprise products such as those of salesforce.com.\autocite[45]{TheAppEconomy}

In many regards cloud computing is a return to more of a dumb terminal paradigm where most of the actual processing takes place in remote data centers rather than on the user's workstation.\autocite[]{LargeParadigmShiftCloudComputing}
