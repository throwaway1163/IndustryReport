\section*{Historical Background}

Early software was usually bundled with hardware as a single integrated product.\autocite[4]{buxmann2012software}

Historically computing was dominated by mainframes like those sold and supported by IBM; in this model all the actual computing is carried out in central processors and the access points to control the computers were dumb terminals.\autocite[]{LargeParadigmShiftCloudComputing}
The personal computer broke this model and pushed most of the industry towards the client/server architecture where much of the processing takes place directly on the users terminal.\autocite[]{LargeParadigmShiftCloudComputing}


Larry Ellison started Software Development Laboratories, Inc. in 1977 with his partners Robert Miner and Edward Oates.\autocite[113]{finkle2012larry}
They changed the company name to Relational Software, Inc. and then to Oracle.\autocite[113]{finkle2012larry}
Oracle was one of the first companies to begin selling relational database management systems, two years before IBM introduced their own systems.\autocite[113]{finkle2012larry}
Oracle's profitability grew rapidly and by 1982 the company had 24 employees and \$2.5 million annual revenue.\autocite[113]{finkle2012larry}
Oracle went public in 1986 and Oracle reached revenue of \$55 million.\autocite[113]{finkle2012larry}
In 1990 Oracle began to post losses and saw its market capitalization fall by 80\%.\autocite[113]{finkle2012larry}
Ellison refocused the company, fired many of the senior managers and replaced them with more experienced hires while he focused more upon product development.\autocite[113]{finkle2012larry}
The Oracle 7 database management system released in 1992 took the industry by storm as Oracle's profits shot up by 76\% within a single quarter.\autocite[114]{finkle2012larry}
Oracle grew rapidly and began to acquire many other business software vendors, spending more than \$25 billion in just 36 months.\autocite[114]{finkle2012larry}
The software industry was a labour intensive business with few barriers to entry.\autocite[115]{finkle2012larry}
The engineers and developers working for Oracle were essential for building the company's software products and maintaining its support contracts.\autocite[115]{finkle2012larry}
The high profit potential of the software business tends to attract many new entrants.\autocite[]{finkle2012larry}
Oracle maintains its strong position in the market partly by aggressively buying out smaller competitors.\autocite[]{finkle2012larry}

The family of operating systems commonly referred to as Linux trace much of their roots back to the GNU project, started by Richard Stallman in 1983 with his essay the GNU manifesto.\autocite[]{GNUManifesto}
In it Stallman declared his intention to create a free operating system denouncing the current trend towards proprietary software development which he believed to be unethical.\autocite[]{GNUManifesto}
The system was to be modeled upon and compatible with the older UNIX, but built up entirely of freely licensed software(GNU is actually a recursive acronym sanding for GNU is Not Unix).\autocite[]{GNUManifesto}

In 1995 Netscape Navigator had more than 80\% market share. However, when Microsoft decided to enter the market it was able to leverage its deep pockets and control of the windows platform to invest in and distribute the Internet Explorer browser. By 1998 Netscape's market share had collapsed falling to bellow 4\% and the navigator browser was open-sourced, creating the Mozilla project. Without competition the Internet Explorer project stagnated while Mozilla continued to develop after being embraced by the open-source community and eventually reemerged as firefox in 2004 with far more features than Internet Explorer and was able to claim significant market share.\autocite[27]{buxmann2012software}

From 1997 to 2000 there was a buying frenzy in the business software industry as the Y2K panic drove greater investment in enterprise software.\autocite[]{AftermathOfIntegratedStack}
This enthusiasitic purchasing diminshed somewhat as business struggled to integrate and make effective use of their investments.\autocite[]{AftermathOfIntegratedStack}
Enterprise software has had a substantial impact upon business over the past few years as these integration efforts have begun to pay off. Retail for example has seen historically low inventories and fewer seasonal hires, arguably as a result efficiencies gained through the effective use of enterprise resource management software. \autocite[]{AftermathOfIntegratedStack}
Companies have now shifted towards what are called integrated stacks, software packages that combine as many of the essentially components of the computer system's architecture as possible. Combining diverse tools into a single product that unifies hardware, middleware, virtualization layer, operating system, and enterprise applications.\autocite[]{AftermathOfIntegratedStack}

The rise of smartphones has seen the creation of App Stores turning most popular mobile devices into software distribution platforms.\autocite[]{TheAppEconomy}
These app stores have seen some remarkable successes as app makers producing everything from small games to enterprise products such as those os salesforce.com.\autocite[45]{TheAppEconomy}


%Other Stuff:

\fig{parametersOfPricing}{h}{.4}{Parameters of pricing models\autocite[83]{buxmann2012software}}{fig:parametersOfPricing}

In many regards cloud computing is a return to more of a dumb terminal paradigm where most of the actual processing takes place in remote data centers rather than on the user's workstation.\autocite[]{LargeParadigmShiftCloudComputing}
