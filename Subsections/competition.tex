The software industry is often identified as an unusally dynamic and competive industry, but this tends to be an oversimpliciation. Several large companies have manage to carve out dominant positions within the industry, such as Google with search, Microsoft with windows, Oracle with enterprise software, and Facebook with social media.\autocite[34]{IndustrySurveysInternet}
Network effects play a large role in creating the ``winner takes it all'' environment, the more customers a given software provider has, the harder it is to dislodge their market position.\autocite[21]{buxmann2012software}
Inter-enterprise network effects influence the market for ERP systems; for example standard formats simplify the exchange of documents between different formats.\autocite[21]{buxmann2012software}

\wrapfig{networkEffects}{r}{.5}{Positive feedback loop}{\Autocite[22]{buxmann2012software}}
Although legacy companies dominate corporate and consumer software, the constant evolution of the technology landscape allows newcomers to be viable competitors.\autocite[38]{SurveysSoftware2015}
Of the 43 companies listed in the S\&P 500 software industry, not a single one was founded after 2000.\autocite[38]{SurveysSoftware2015}
However the four biggest software companies listed, Microsoft, IBM, Oracle, and SAP together only accounted for 39\% of industry revenues.\autocite[38]{SurveysSoftware2015}

Indirect network effects occur when a given software product leads to a body of associated complimentary software that depends upon it.
This essentially increases the cost of competing with it as the complimentary software must be replaced along with the primary platform itself.\autocite[21]{buxmann2012software}

Mergers and acquisitions are an important part of business in any industry and software is no exception.\autocite[421]{schief2013mergers}
M\&As tend to come in waves focused on particular industries; especially when firms are experiencing high liquidity within a profitable industry.\autocite[421]{schief2013mergers}
The software industry is unusually prolific with regards to M\&A ranking second out of 49 industries in the United States.\autocite[421]{schief2013mergers}
Small companies frequently get off to a strong start with venture capital but they are often bought up by their larger rivals once they have grown to the point where they can begin to make themselves a nuisance. \autocite[]{LargeParadigmShiftCloudComputing}\autocite[]{ValuelineOverview}
Oracle grew rapidly and began to acquire many other business software vendors, spending more than \$25 billion in just 36 months.\autocite[114]{finkle2012larry}%Justify the inclusion of oracle
Oracle maintains its strong position in the market partly by aggressively buying out smaller competitors.\autocite[121]{finkle2012larry}

\wrapfig{networkEffectGraph}{r}{.4}{Greater utility due to greater adoption}{\autocite[23]{buxmann2012software}}
Since network-effects tend to play a significant role in the software industry, there is strong pressure for firms to consolidate in order to gain greater competitive advantage over their rivals.\autocite[422]{schief2013mergers}
In the long term this means that software markets tend to develop oligopolistic or monopolistic structures with most of the market share and power focused within just a few large firms.\autocite[422]{schief2013mergers}
The nature of the software industry is to tend towards greater consolidation.\autocite[]{LargeParadigmShiftCloudComputing}
Software markets tend to eventually create monopolies, that is to say that the first firm to gain a sufficiently large competitive advantage will usually eventually eradicate its competition.\autocite[4]{buxmann2012software}
Microsoft is a good example of a company that quickly rose to monopolistic heights with MS-DOS, coming to almost completely dominate the desktop market.\autocite[4]{buxmann2012software}

\wrapfig{increasingReturns}{r}{.4}{Path dependent outcome}{\autocite[25]{buxmann2012software}}
Since the software industry is not dependent on any material resources, it is not beholden to powerful suppliers.
However, the S\&P Capital IQ suggest that software companies instead compete over another type of limited resource -- skilled developers -- and that smaller companies are at a disadvantage with regards to acquiring and retaining capable employees.\autocite[39]{SurveysSoftware2015}
Larger companies can attract more talented employees, invest more in research and developing new products, as well as buy out smaller rivals; thus applying further pressure on the market to consolidate.\autocite[34]{IndustrySurveysInternet}

It is a widely accepted economic fact that one of the most important ways a firm can gain competitive advantage is through innovating and introducing new products.
These gains are, however, generally limited and temporary as competitors reuse and refine these ideas for their own benefit.\autocite[81]{ceccagnoli2007appropriability}
The efficacy of patents in promoting a firm's profitability is up for debate and recent literature suggests that patents carry significant value for their holding firms.\autocite[]{ceccagnoli2007appropriability}
The usefulness of a patent can be described in terms of the extent to which it increases the appropriability of a firms innovations.\autocite[81]{ceccagnoli2007appropriability}
Appropriability refers to how high a portion of the value created by an innovation can be captured by a firm.\autocite[82]{ceccagnoli2007appropriability}
Erecting some kind of barrier around core business technologies can dramatically improve a company's long-term prospects.\autocite[]{HorizontalPlaysTechnology}
This sort of protection can be achieved through patents, network effects, or differentiation, but without some sort of barrier a company's business will remain vulnerable with the ever changing landscape of the software industry.\autocite[]{HorizontalPlaysTechnology}
Companies can also engage in preemptive patenting, aimed at trying to blockade potential competitors from infringing on their market.\autocite[83]{ceccagnoli2007appropriability}

Open source software is also a significant part of the competitive landscape of the software industry.
There were almost one billion lines of source code released publicly and freely available online in 2010 offering a compelling alternative to propeitary solutions in many software segments.\autocite[]{OptimismInnovation}
Some industry analysts believe that the open source development process ultimately leads to better code than proprietary alternatives.\autocite[]{OptimismInnovation}
Projects are sustained by an international community of volunteer developers who pool their efforts to produce software.\autocite[191]{buxmann2012software}
Many open source projects are controlled by their founder who usually functions as the maintainer and retains the authority to decide whether or not to add other developers' work to the mainline project.
If the community grows dissatisfied with the maintainer, the project can be forked (the publicly available source code is copied to form a new project) and development continues under a new banner.
Since the individual developers are volunteers who decide how to spend their own time small projects start and stop frequently as their developers move on to other things.\autocite[197]{buxmann2012software}
The opportunity costs in terms of time and effort inherent to contributing effort towards Open source are obvious, which raises the question of why anyone would do it.
Most contributers fall into one of two broad catagories, rent-seekers who work in the hopes of raising their profile, gaining experience, and adding lines to the resume, and donors who work to contribute to the strength of the open source ecosystem.\autocite[198]{buxmann2012software}

\wrapfig{kernelDevelopment}{r}{0.5}{Top contributers to the 4.3 Linux Kernel}{\autocite[]{KernelDevelopment2015}}
Open source projects are often published using ``copyleft'' licenses such as GPL (GNU General Public License) which require that derivative works source code always be made available and for any to be published under similar terms.
This creates an inducement for donors to contribute code since they can be confident that their work will remain open source, and it forces organizations that make use of the work to push their contributions and changes back to the community.\autocite[198]{buxmann2012software}
By forcing users to contribute back to the project these copyleft licenses also help avert the so called ``tragedy of the commons'' where individual rational business would prefer to control as much of their intelectual property as possible, allowing the original project -- itself a shared resource -- to grow obsolete and unmaintained.
Benefit from the intelectual proproperty of the project but have no incentive to contribute to a shared resource that could potentially benefit their competitors in the future.
Large projects are usually sustained not by volunteers but by for-profit business seeking to add their own extensions to an existing project for their own benefit.
In fact, work by individual contributors to famous projects such as the Linux kernel is at an all time low with only 5.8\% of the changes in the past year coming from unpaid volunteers.\autocite[]{KernelDevelopment2015}

%Change!!!!!!!!!!!!!!!!!!!!!!!!

Red Hat is another interesting player given that they have long been at the center of the open source movement and have managed to act like a sort of curator for open source technology.\autocite[]{OptimismInnovation}
Red Hat has built their entire business around software that can be freely copied by their competitors, and as a result has minimal switching costs.
Yet Red Hat's revenues continue to grow considerably faster than overall IT spending.\autocite[]{OptimismInnovation}

\iffalse
\begin{minipage}{\textwidth}
\begin{tabu}{| l | c | c | c | c | c |} \hline
    Market & Linux Kernel Based & Unix \& BSD & Windows & iOS \& OS X \\ \hline
    Supercomputers\autocite{Top500} & 98.8\% & 1.2\% & N/A & N/A \\ \hline
    Device Shipments\autocite{Gartner} & 1,156,111 (Android) & N/A & 333,358 & 262,615 \\ \hline
    Web Servers\autocite{W3Cook} & 96.3\% & 1.7\% & 1.8\% &  N/A \\ \hline
    Web Servers\autocite{W3Techs} & N/A & 67.8\% \emph{incl. linux} & 32.2\% & N/A \\ \hline
\end{tabu}
\end{minipage}
\fi

\ref{fig:webServerMarketShare}
