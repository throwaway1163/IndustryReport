\section*{Competition}

The software industry is often identified as an unusally dynamic and competive industry, but this tends to be an oversimpliciation. Several large companies have manage to carve out dominant positions within the industry, such as Google with search, Microsoft with windows, Oracle with enterprise software, and Facebook with social media.\autocite[34]{IndustrySurveysInternet}
\wrapfig{softwareIndustryMarketCap}{r}{.5}{Market capitalization data\autocite[21]{SurveysSoftware2015}}{fig:softwareIndustryMarketCap}
\wrapfig{ITMarketCap}{r}{.5}{Market capitalization data\autocite[6]{SurveysSoftware2015}}{fig:ITMarketCap}
Network effects play a large roll in creating the ``winner takes it all'' environment, the more costumers a given software provider has the harder it is to dislodge their market position.\autocite[21]{buxmann2012software}
Inter-enterprise network effects influence the market for ERP systems, for example standard formats simplify the exchange of documents between different formats.\autocite[21]{buxmann2012software}

Although legacy companies dominate corporate and consumer software the constant evolution of the technology landscape allows newcomers to be viable competitors.\autocite[38]{SurveysSoftware2015}
Of the 43 companies listed in the S\&P 500 software industry not a single one was founded after 2000.\autocite[38]{SurveysSoftware2015}
The four companies with the largest software business Microsoft, IBM, Oracle, and SAP together only acounted for 39\% of the revenues in the industry.\autocite[38]{SurveysSoftware2015}

Indirect network effects are when a given software product leads to a body of associated complimentary software that depends upon it, this essentially increases the cost of competing with it as the complimentary software must be replaced along with the primary platform itself.\autocite[21]{buxmann2012software}
\wrapfig{networkEffects}{r}{.5}{Positive feedback loop\autocite[22]{buxmann2012software}}{fig:networkEffects}
\wrapfig{networkEffectGraph}{r}{.4}{Greater utility to the consumer as a result of greater adoption\autocite[23]{buxmann2012software}}{fig:networkEffectGraph}
\wrapfig{increasingReturns}{r}{.5}{Path dependent outcom of competition in a market influenced by network effects.\autocite[25]{buxmann2012software}}{fig:increasingReturns}

Mergers and Acquisitions are an important part of business in any industry and software is no exception.\autocite[421]{schief2013mergers}
M\&As tend to come in waves focused upon particular industries usually when firms are experiencing high liquidity within a profitable industry.\autocite[421]{schief2013mergers}
In fact the software industry is unusually prolific ranking second out of 49 industries in the United States.\autocite[421]{schief2013mergers}
Small companies are often able to get off to a strong start due to venture capital but they are often bought up by their larger rivals once they have grown to the point where they can begin to make themselves a nuisance. \autocite[]{LargeParadigmShiftCloudComputing}

Since network-effects tend to play a significant role in the software industry their is strong pressure for firms to consolidate in order to gain greater competitive advantage over their rivals.\autocite[422]{schief2013mergers}
In the long term this means that software markets tend to develop oligopolistic or monopolistic structures with most of the market share and power focused within just a few large firms.\autocite[422]{schief2013mergers}
The nature of the Software industry is to tend towards greater consolidation.\autocite[]{LargeParadigmShiftCloudComputing}
Software markets tend to eventually create monopolies, that is to say that the first firm to gain a sufficiently large competitive advantage will usually eventually eradicate it's competition.\autocite[4]{buxmann2012software}
Microsoft is a good example of a company that quickly rose to monopolistic heights with ms-dos, coming to almost completely dominate the desktop market.\autocite[4]{buxmann2012software}

There are many new entrants in the software industry, and it is comparatively easy for software start ups gain large quantities funding and high early valuations. However building up revenue and establishing stable relations with clients and sustainable profits takes much more time.\autocite[39]{SurveysSoftware2015}
The industry currently engaged in a transition to cloud based services.\autocite[39]{SurveysSoftware2015}
Since the software industry is not dependant upon any material resources it is not beholden to its suppliers. However the S\&PCapital IQ suggest that software companies instead compete over another type of limited resource, skilled developers; and that smaller companies are at a disadvantage with regards to acquiring and retaining capable employees.\autocite[39]{SurveysSoftware2015}
Larger companies can attract more talented employees, invest more in research and developing new products, as well as buy out smaller rivals; thus applying pressure on the market to consolidate.\autocite[34]{IndustrySurveysInternet}

It is widely accepted economic fact that one of the most important ways a firm can gain competitive advantage is through innovating and introducing new products. However these gains are generally limited and temporary as competitors reuse and refine these ideas for their own benefit.\autocite[81]{ceccagnoli2007appropriability}
The efficacy of patents in promoting a firm's profitability is up for debate and recent literature suggests that patents carry significant value for their holding firms.\autocite[]{ceccagnoli2007appropriability}
The usefulness of a patent can be described in terms of the extent to which it increases the appropriability of a firms innovations.\autocite[81]{ceccagnoli2007appropriability}
Appropriability refers to how high a portion of the value created by an innovation can be captured by a firm.\autocite[82]{ceccagnoli2007appropriability}
Erecting some kind of barrier around core business technologies can dramatically improve a company's long-term prospects.\autocite[]{HorizontalPlaysTechnology}
This sort of protection can be achieved through patents, network effects, or differentiation, but without some sort of barrier a company's business will remain vulnerable with the ever changing landscape of the software industry.\autocite[]{HorizontalPlaysTechnology}
Companies can also engage in preemptive patenting, aimed at trying to blockade potential competitors from infringing upon their market.\autocite[83]{ceccagnoli2007appropriability}

Open source projects operate very differently from the traditional for profit centrally directed organizational model of commercial software. Open Source projects are sustained by an international pool of volunteer developers who pool their efforts to produce software.\autocite[191]{buxmann2012software}
Many open source projects are controlled by their founder who usually functions as the maintainer and retains the authority to decide whether or not to add other developer's work to the mainline project. Linus for example has the final say in whether or not to incorporate changes other's make to the Linux kernel. If the community grows dissatisfied with the maintainer the project can be forked and development continues under a new banner. Since the individual developers are volunteers who decide how to spend their own time small projects start and stop frequently as their developers move on to other things.\autocite[197]{buxmann2012software}
The opportunity costs in terms of time and effort inherent to contributing effort towards Open Source are obvious, which raises the question of why anyone would do it. Most contributers fall into one of two broad catagories, rent-seekers who work in the hopes of raising their profile, gaining experience, and adding lines to the resume, and donators who work to contribute to the strength of the Open Source ecosystem.\autocite[198]{buxmann2012software}
Open Source projects are often published under copyleft licenses such as GPL which require the source code to always be made available and for any derivatives works to be published under similar terms. This creates an inducement for donators to contribute code since they can be confident that their work will remain open source, and it forces organizations that make use of the work to push their contributions and changes back to the community.\autocite[198]{buxmann2012software}
Work by individual contributors to the Linux kernel is at an all time low with only 5.8\% of the changes coming from unpaid volunteers.\autocite[]{KernelDevelopment2015}
\wrapfig{kernelDevelopment}{r}{0.5}{Top contributers to the 4.3 Linux Kernel\autocite[]{KernelDevelopment2015}}{fig:kernelDevelopment}
Red Hat is another interesting player given that they have long been at the center of the open source movement and have managed to act like a sort of curator for open source technology.\autocite[]{OptimismInnovation}
There were almost one billion lines of source code released publicly and freely available online in 2010.\autocite[]{OptimismInnovation}
Red Hat has built there entire business around software that can be freely copied by their competitors, and as a result has minimal switching costs. Yet Red Hat's revenues continue to grow considerably faster than overall IT spending.\autocite[]{OptimismInnovation}
Some industry analysts believe that the open source development process ultimately leads to better code than proprietary alternatives.\autocite[]{OptimismInnovation}

%Other Stuff:

\wrapfig{TopInternetRevenues}{r}{.5}{The top few players dominate the internet software sector\autocite[35]{IndustrySurveysInternet}}{fig:TopInternetRevenues}
