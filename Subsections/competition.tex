One of the most important characteristics of software is it's reproducibility.\autocite[3]{buxmann2012software}
Since variable costs involved in software redistribution are effectively negligable it would seem natural to conclude that software licensing is more profitable than providing software services, which do not scale as cleanly.\autocite[3]{buxmann2012software}
The software market is also highly international in nature and software products are often developed by distributed teams.\autocite[3]{buxmann2012software}
It is widely accepted economic fact that one of the most important ways a firm can gain competitive advantage is through innovating and introducing new products. However these gains are generally limited and temporary as competitors reuse and refine these ideas for their own benefit.\autocite[81]{ceccagnoli2007appropriability}
The efficacy of patents in promoting a firm's profitability is up for debate and recent literature suggests that patents carry significant value for their holding firms.\autocite[]{ceccagnoli2007appropriability}
The usefulness of a patent can be described in terms of the extent to which it increases the approachability of a firms innovations.\autocite[81]{ceccagnoli2007appropriability}
Appropriability refers to how high a portion of the value created by an innovation can be captured by a firm.\autocite[82]{ceccagnoli2007appropriability}
Companies can also engage in preemptive patenting, aimed at trying to blockade potential competitors from infringing upon their market.\autocite[83]{ceccagnoli2007appropriability}
The software industry can be divided into two broad segments, business derived from the sale of software licenses, and the business of selling software and services by contract.\autocite[]{ValuelineOverview}
It is relatively easy for new entrants to break into the market, though given it's competitive nature they often struggle to maintain their edge and are often acquired by larger rivals if they do not grow quickly enough.\autocite[]{ValuelineOverview}
There are two forces that tend to drive spending in the enterprise software industry, there are IT-driven initiatives and there are business driven initiatives.\autocite[]{NextGenBusinessSoftware}
Business driven IT developments tend to be outward facing and focus upon concrete business problems rather than underlying technologies, and in it is this sort of spending that drives much of current IT investment.\autocite[]{NextGenBusinessSoftware}
There has been a major shift towards using software to help analyze data in order to make decisions and draw insights that would otherwise be beyond the reach of a human observer.\autocite[]{NextGenBusinessSoftware}
The capacity to extract useful information from huge piles of data, such as tweets or facebook posts.\autocite[]{NextGenBusinessSoftware}
Naturaly privacy is a big concern, but many modern technology platforms operate upon the assumption that the user doesn't care about privacy at all. Services like free email, cloud storage, and social networking almost invariably require the end user to give up any real expectation of privacy. Moreover many of these business are directly dependent upon gathering their users data in order to turn a profit.\autocite[]{NextGenBusinessSoftware}
Microsoft is currently encouraging users to shift from Office  to Office 365, their cloud based alternative.\autocite[]{NextGenBusinessSoftware}
