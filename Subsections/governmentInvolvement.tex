Government involvement is for the most part not a particularly important part of the software industry.
There have been some significant antitrust cases within the industry, most notably the case brought against Microsoft in 2001, but most did not end in any convictions or breakups.
Indeed the judgement was seen as little more than a slap on the wrist, analyst Robert Cringely joked that the only way Microsoft could die is by suicide.\autocite{ImmortalMicrosoft}

One of the few significant areas of government regulation in the software industry pertain to privacy and encryption with respect to concerns about national security, especially in light of current terrorist threats.\autocite[]{TerroristData}
The software industry remains strongly against regulating encryption, at least in public, in light of public concerns about government spying.\autocite[]{TerroristData}
It is also obvious that simply banning encryption outright would be as disastrous as passing legislation against roads since so much digital Commerce and communication depends upon strong encryption.\autocite[]{TerroristData}
However surveillance advocates argue that it may be necessary to undermine public encryption standards in the name of increased surveillance, in spite of its potential economic cost (naturally there are considerable legal and constitutional issues with mass-surveillance as well, but these are beyond the scope of this paper).
Privacy concerns are particularly contentious when issues of national borders are raised. The recent Snowden revelations in particular have prompted interest in trying to keep data currently stored in data centers outside the United States from being sent back to the US, and presumably to the National Security Agency as well.\autocite[]{OffMyCloud}
The government has made efforts to curb the export of technology deemed to be of use to potential enemies of the state; the basic argument is that trade in modern software is the 21st century equivalent of the arms trade.\autocite[]{HeatsUpSurveillance}
The commerce Department proposed new licensing regulations in May for software that can be useful in surveillance. This of course raised fears about the vague nature of the proposal given that a great quantity of software can have some sort of dual purpose.\autocite[]{HeatsUpSurveillance}

This sort of interference is widely believed to have gone even further than has already been publicly disclosed.
By some accounts national security agencies may have actually been directly weakening encryption protocols in order to facilitate data collection.\autocite[]{CrackedCredibility}
This undermines the web of trust that underpins the internet and allows for everything from secure communications to online banking.\autocite[]{CrackedCredibility}
The NSA has been known to make highly suspect recommendations when it comes to cryptographic technology. In 2006 the agency promoted a random number-generating device that performed terribly and was later found to suffer from serious security flaws.\autocite[]{CrackedCredibility}
Back doors such as these are risky and controversial since they could potentially be exploited by anyone, not just intelligence agencies, and of course privacy advocates would not be satisfied with having anyone spy on private communications.\autocite[]{CrackedCredibility}
American software firms have suffered as a result of Snowden's disclosures regarding the extent of America's spying operation as it fuels protectionism abroad and has led foreign business to regard American companies with suspicion and potential losses measure in the tens of billions.\autocite[]{CrackedCredibility}
As a result businesses in the software industry have a strong incentive to oppose surveillance as loudly and as publicly as possible so as to reassure their customers.
Microsoft for example has been fighting a recent search warrant for data stored in an Irish data center were Microsoft claims it is under the jurisdiction of European privacy laws.\autocite{MicrosoftEmail}
Apple CEO Tim Cook has also taken a stand on encrypted communication, publicly declaring his opposition to any efforts to insert back doors into encryption protocols.\autocite[]{StrongEncryptionCook}
The issue of government surveillance is unlikely to be resolved in the near future as surveillance activity shows no signs of slowing down.
Recently Google was allowed to publish general statistics on the scale of surveillance activity, and has reported a 45\% rise in the number of accounts that security agencies have requested access to over the past year.\autocite[]{DataRequests}
