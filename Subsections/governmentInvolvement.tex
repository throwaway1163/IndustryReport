One of the most significant and talked about aspects of government regulation of the software industry pertains to encryption and concerns about national security in light of current terrorist threats.\autocite[]{TerroristData}
The software industry remains strongly against regulating encryption, at least in public, in light of the public concerns about government spying.\autocite[]{TerroristData}
It is also obvious that simply banning encryption outright would be be as disastrous as passing legislation against roads.\autocite[]{TerroristData}
Privacy concerns are particularly contentious when issues of national borders are brought up. The recent Snowden revelations in particular have prompted interest in trying to keep data currently in data centers outside of the United States from being sent back the US, and presumably to the NSA as well.\autocite[]{OffMyCloud}
The government has made efforts to curb the export of technology deamed to be of use to potential enemies of the state, the basic argument is that trade in modern software is the 21st century equivalent of the arms trade.\autocite[]{HeatsUpSurveillance}
The commerce Department proposed new licensing regulations in May for software that can be useful in surveillance. This of course raised fears about the vague nature of the proposal given that a great quantity of software can have some sort of dual purpose.\autocite[]{HeatsUpSurveillance}

By some accounts national security agencies may have actually been directly undermining encryption protocols in order to facilitate data collection.\autocite[]{CrackedCredibility}
This undermines the web of trust that underpins the internet and allows for everything from secure communications to online banking.\autocite[]{CrackedCredibility}
The NSA has been known to make highly suspect recommendations when it comes to cryptographic technology. In 2006 the agency promoted a random number generating device that suffered terrible performance and was latter found to suffer from serious security flaws.\autocite[]{CrackedCredibility}
Backdoors such as these are risky and controversial since they could potentially be exploited by anyone, not just intelligence agencies, and of course privacy advocates would not be satisfied with having anyone spy on private communications.\autocite[]{CrackedCredibility}
American software firms have suffered as a result of Snowden's disclosures regarding the extent of American's spying operation as it fuels protectionism abroad and has led foreign business to regard American companies with suspicion with potential losses measuring in the tens of billions.\autocite[]{CrackedCredibility}
Google has seen a 45\% rise in the number of accounts security agencies have requested access to over the past year.\autocite[]{DataRequests}
Apple CEO Tim Cook has taken a strong stance on encrypted communication publicly declaring his opposition to any efforts to insert back doors into encryption protocols.\autocite[]{StrongEncryptionCook}
