The software industry is expected to grow in the coming years as software grows more powerful and abundant.\autocite[21]{SurveysSoftware2015}
There will likely be another reshuffling of the industry as more companies and consumers begin to use more cloud-based and products and the software as a service model gains more momentum.
Oracle faces competition both from large vendor's such as IBM and Microsoft that offer broadly similar enterprise software products, as well as smaller rivals offering substitute services.\autocite[]{finkle2012larry}
Oracle dominates the relational database management systems business with 48\% market share, far more than any of its rivals.\autocite[]{finkle2012larry}
Cloud based Software as a Service companies like Salsesforce.com can provide powerful business tools with relatively low administrative and licensing costs, which has helped enable salesforce.com's robust 40\% annual growth.\autocite[]{HorizontalPlaysTechnology}
Amazon has seen exponential growth in recent years and has become a juggernaut in the retail world, but has done so at the expense of profits.\autocite[]{AmazonProfit}
Amazon has held onto a sliver of profit, having only lost money in two out of the last 10 fiscal years. However it's earnings have been abysmal, peaking at \$2.53 per share in 2010.\autocite[]{AmazonProfit}
Many analysts are bullish on amazon, even after its stock more than doubled over the past year.\autocite[]{AmazonDouble}
Citi Research expects Amazons earnings per share to grow to \$17.53 in 2018.\autocite[]{AmazonProfit}
Amazon's cloud business, Amazon Web Services, currently holds 48\% market share, a figure Citi Research expects will grow to 68\% by 2018 with expected revenues reaching \$24 billion from \$8 billion.\autocite[]{AmazonProfit}

Although the industry is relatively dynamic it is not actually as volatile or competitive a market as is often assumed.
Industry giants tend to be difficult to dislodge and actually have quite a few competitive advantages over startups.
Indeed, although their is still substantial room for growth, the technology industry is arguably at a stage of relative maturity with corporations like Oracle and Microsoft commanding immense market capitalizations.\autocite[]{finkle2012larry}

However, while these transitions will likely result in significant changes in industry leadership companies with older business models or products are by no means doomed.
Indeed established software companies and products tend to survive for decades after ``going out of fashion,'' even mainframes are still a multi-billion dollar industry hosting mission critical applications in hundreds of large business.\autocite{MainframesStillAround}
The enterprise software market is plagued by high switching costs,\autocite[]{finkle2012larry} and new technologies are not necessarily inherently superior.

Overall software remains a significant growth sector with plenty of opportunities for expansion and competition.
It certainly is an interesting and worthwhile area for investment and will remain so for the foreseeable future.
