\section*{Conclusion}

Oracle faces competition both from large vendor's such as IBM and Microsoft that offer broadly similar enterprise software products, as well as smaller rivals offering substitute services.\autocite[]{finkle2012larry}
Although their is still substantial room for growth, the technology industry is arguably at a stage of relative maturity with industry giants like Oracle and Microsoft commanding immense market capitalizations.\autocite[]{finkle2012larry}
Their exists high switching costs in the market for enterprise software solutions.\autocite[]{finkle2012larry}
Oracle dominates the relational database management systems business with 48\% market share, far more than any of its rivals.\autocite[]{finkle2012larry}
Cloud based Software as a Service companies like Salsesforce.com can provide powerful business tools with relatively low administrative and licensing costs, which has helped enable salesforce.com's robust 40\% annual growth.\autocite[]{HorizontalPlaysTechnology}


Software companies are often valued based upon forward revenue leading to inflated price to earnings ratios, and particularly popular firms such as Tableau, an analytics firm, can have high valuations even with respect to forward revenue.\autocite[]{HorizontalPlaysTechnology}
\wrapfig{normalizedPE}{r}{.8}{Forward and normalized price earnings to ratio}{\autocite[27]{SurveysSoftware2015}}

Many analysts are bullish on amazon, even after its stock more than doubled over the past year.\autocite[]{AmazonDouble}
Amazon has seen exponential growth in recent years and has become a juggernaut in the retail world, but has done so at the expense of profits.\autocite[]{AmazonProfit}
Amazon has held onto a sliver of profit, having only lost money in two out of the last 10 fiscal years. However it's earnings have been abismal, peaking at \$2.53 per share in 2010.\autocite[]{AmazonProfit}
Citi Research expects Amazons earnings per share to grow to \$17.53 in 2018.\autocite[]{AmazonProfit}
Amazon's cloud business, Amazon Web Services, currently holds 48\% market share, a figure Citi Research expects will grow to 68\% by 2018 with expected revenues reaching \$24 billion from \$8 billion.\autocite[]{AmazonProfit}

The software industry is expected to grow in the coming years as software grows more powerful and abundant.\autocite[21]{SurveysSoftware2015}
