Overall IT spending has been growing at an annual rate of about 2-3\%, while cloud-based software, big data analytics, and cybersecurity are expected to remain growth areas.\autocite[]{CloudAnalyticsCybersecurityApplication}\hspace{5mm}\autocite[]{CIOIntro}
The public cloud market is expected to increase from \$58bn in 2013 to \$101bn by 2020, with the market for cloud applications growing to \$133bn over the same period.\autocite[]{CIOIntro}
Cloud systems tend to be more scalable and easy to use than in house solutions.\autocite[]{CIOIntro}

\wrapfig{normalizedPE}{r}{.5}{Forward and normalized price earnings to ratio}{\autocite[27]{SurveysSoftware2015}}
Software companies are often valued based upon forward revenue leading to inflated price to earnings ratios, and particularly popular firms such as Tableau, an analytics firm, can have high valuations even with respect to forward revenue.\autocite[]{HorizontalPlaysTechnology}

The industry is currently engaged in a transition to cloud-based services.\autocite[39]{SurveysSoftware2015}
Cloud computing is really just a model of renting storage and processing time to clients.\autocite[]{AftermathOfIntegratedStack}
The real advantage of cloud computing is that the host can take advantage of far greater economies of scale than their individual clients could on their own, pushing down costs dramatically.\autocite[]{AftermathOfIntegratedStack}
Integrated stacks can lower the cost of deploying new enterprise software solutions, whereas in the past most of the expense involved in deploying new software was actually in having internal IT departments integrate all of the diverse tools and platforms.
Integrated stacks can be purchased, deployed, and maintained quickly and cheaply.\autocite[]{AftermathOfIntegratedStack}
Cloud service providers also integrate their own tools into their offerings to help their costumers use and manage their cloud resources.\autocite[]{AftermathOfIntegratedStack}
There is a growing shift towards more ``transparent'' IT solutions which try to operate seemlessly with less input from the end user along with more embedded and cloud-based applications.\autocite[]{OptimismInnovation}

Amazon is one example of company that has taken full advantage of the rise of cloud computing.
Amazon's cloud business grew 78\% year over year recently.\autocite[]{GoogleCloudBoost}
Industry analyst Rob Sanderson describes Amazon Web Services (their industry leading cloud hosting solution) as being ``so robust with rich services, most of which are products of iterative enhacement that I think it will be very difficult for any company, including Google to catch up.''\autocite[]{GoogleCloudBoost}

Microsoft's emphasis on their Azure cloud platform represents a major shift for the company as it tries to push its way into an already competitive and crowded field.\autocite[]{OptimismInnovation}
Microsoft is currently encouraging users to shift from Office  to Office 365, their cloud-based alternative.\autocite[]{NextGenBusinessSoftware}
Vmware is another signficant player in the data-center business, focusing on virtualization technologies that are becoming increasingly common in these sorts of environments.\autocite[]{NextGenBusinessSoftware}
Virtualization is another important technology that has begun to play a significant role over the past few years.\autocite[]{LargeParadigmShiftCloudComputing}
Virtualization allows multiple operating systems and software stacks to be hosted on a server by introducing a layer of abstraction between the operating systems and the raw hardware of the host machine.\autocite[]{LargeParadigmShiftCloudComputing}

Software as a service is a growing component of the software industry and holds several key advantages over more traditional sales models, most importantly in terms of cost and ease of deployment.\autocite[]{GrowthTrendsSoftwareService}\autocite[]{TransitionPhaseEnterpriseApplication}
The computer service and support industry is heavily dependent on the relationship of trust between the service provider and the consumer, who usually have to base their decision of references.\autocite[9]{buxmann2012software}
The rise of cloud computing has made it far easier for the consumers of enterprise software products to try out different products and shop between competing vendors before making a potentially risky deal.\autocite[29]{LiquidLunch}
As a result vendors are forced to compete more on substance.\autocite[29]{LiquidLunch}.
This change has been reflected among the ranks of salesmen who have seen a dramatic growth in wage disparity between the most and least successful representatives -- those with strong technical knowledge of their products have come into much higher demmand.\autocite[30]{LiquidLunch}

Naturally this shift towards cloud computing is not the only way foward within the industry and many companies have either chosen to or been unable to transition their services to be competitive within the new market.
Notably, Oracle has decided not to follow Amazon in offering raw cloud computing and data storage services.
Oracle co-CEO Mark Hurd claimed that this decision was based on the low margins of raw services like Amazon's.\autocite[]{OracleCloudNotAWS}
Oracle largely missed the bus in the cloud hosting business given that it would likely cost billions of dollars to acquire the talent, resources, and know-how in order to compete with the current industry leaders.\autocite[]{OracleCloudNotAWS}
However both SAP and Oracle have begun to ramp up their cloud-based services by choosing to build on top of the AWS platform.\autocite[]{OracleCloudNotAWS}
Amazon has not yet invested in trying to develop enterprise applications hosted on its own cloud infrastructure.

Oracle co-CEO Mark Hurd recently reported that ``We now have virtually 100\% of our portfolio rewritten, rebuilt and modernized for the cloud.''\autocite[]{OracleRebuilt}
Cloud-based services have seen rapid adoption: Oracle added 200 new ERP customers over the last quarter and has drawn approximately \$5 billion dollars in research and development investment.\autocite[]{OracleRebuilt}
However in spite of this growth Oracle's cloud-based business offerings still only account for 7\% of total revenue.\autocite[]{OracleNoGrowth}
All the while, Oracle has seen its other business stagnate and decline.\autocite[]{OracleNoGrowth}

One expected area of future development in the industry is data analysis, using powerful software to analyze the raw data in order to make decisions and draw insights that would otherwise be beyond the reach of a human observer.\autocite[]{NextGenBusinessSoftware}
This sort of analysis would be dependant upon the huge piles of data, generated through Facebook posts, smartphones, browser activity, search engine querries, and myriads of other sources.\autocite[]{NextGenBusinessSoftware}
Privacy is naturally a big concern, but many modern technology platforms operate on the assumption that the user doesn't care about privacy at all.
Services like free email, cloud storage, and social networking almost invariably require the end user to give up any real expectation of privacy.
Moreover many of these business are directly dependent on gathering their users' data in order to turn a profit.\autocite[]{NextGenBusinessSoftware}
