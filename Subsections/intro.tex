\section*{Introduction}

The software industy is one of the largest and most dynamic in United States, accounting for 19.5\% of the S\&P 500 in July 2015.\autocite[6]{SurveysSoftware2015}
Unlike most other industries that produce tangible goods or provide services dependent upon limited inputs, software is trivially reproducible.\autocite[3]{buxmann2012software}
Furthermore, it is relatively easy for new entrants to enter the market, though given it's competitive nature they often struggle to maintain their edge and are often acquired by larger rivals if they do not grow quickly enough.\autocite[]{ValuelineOverview}
The software market is also highly international in nature and software products are often developed by distributed teams.\autocite[3]{buxmann2012software}
Although the domestic market is still the primary driver of revenue and has continued to see robust growth.\autocite[]{ITSoftwareEconomist}

Their are two primary ways to turn a profit off of software, one time licensing fees, which are most popular with consumer software, and recurring maintenance fees more common in business software.
\fig{genericBusinessModels}{h}{.32}{Generic Business Models\autocite[15]{buxmann2012software}}{fig:genericBusinessModels}
The industry began to make a shift to standardized software early on as producers tried to take advantage of their negligible variable costs.\autocite[14]{buxmann2012software}

The market for business software can be divided further into two broad segments, business derived from the sale of software licenses, and the business of selling software and services by contract.\autocite[]{ValuelineOverview}
Consulting fees are usually proportional to the number of working days hired or results while maintenance service fees are usually a flat fraction of the original licensing fee.\autocite[16]{buxmann2012software}
\wrapfig{marketShareOfTypesOfSoftware}{r}{.5}{Market share of software types\autocite[7]{buxmann2012software}}{fig:marketShareofTypesOfSoftware}
The business model of these standards software providers is usually to sell licenses which charge the customer proportional to some metric of usage\autocite[14]{buxmann2012software}

The field of transaction theory is useful in understanding how and why business choose to externally source their software ~/ref{fig:1}.
Since the software industry is defined by the division of labor between software providers and consumers their should theoretically exist substantial benefits to organizational structures that minimize transaction costs.\autocite[42]{buxmann2012software}
Transactions can be classified along a spectrum according to how specific they are to particular investments.
Markets are the most efficient tool to optimize transaction costs in situations where the specificity of the transactions is low.
On the other hand more specific investments where the transaction partners are more interdependent encourage more opportunistic behavior as the parties have greater incentive to try to exploit each other.\autocite[44]{buxmann2012software}
\wrapfig{decisionMatrix}{r}{.5}{Decision matrix of IT outsourcing\autocite[45]{buxmann2012software}}{fig:DecisionMatrix}
The shift towards the internet and IP-based services has had the effect of lowering transaction costs, therefore encouraging standardization of software from dedicated providers rather than homegrown solutions.\autocite[46]{buxmann2012software}
Since variable costs involved in software redistribution are effectively negligible it usually makes sense to externally source any business related software that is not highly specific to a given company.

There are two forces that tend to drive spending in the enterprise software industry, there are IT-driven initiatives and there are business driven initiatives.\autocite[]{NextGenBusinessSoftware}
Business driven IT developments tend to be outward facing and focus upon concrete business problems rather than underlying technologies, and in it is this sort of spending that drives much of current IT investment.\autocite[]{NextGenBusinessSoftware}
Cloud based software in particular has seen dramatic and sustained growth in recent years, a trend which is expected to continue.
The projected compound annual growth rate of cloud software from 2013 to 2018 currently rests at 20\% indicating a substantial shift within the industry. Significant players such as Adobe, Autodesk, Oracle, and Microsoft have realigned themselves to try take advantage of the ongoing shift towards cloud based services.\autocite[40]{SurveysSoftware2015}
Enterprise resource management is another important sector of the software industry including business focused upon managing accounting, payroll, and human capital.\autocite[46]{SurveysSoftware2015}
The internet has grown dramatically over the past years, which has helped enable many of the most important transitions within the industry including the shift towards cloud based services and the proliferation of open-source software.\autocite[]{SurveysSoftware2015}
