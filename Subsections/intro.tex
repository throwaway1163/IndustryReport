\section*{Introduction}

\wrapfig{ITMarketCap}{r}{.5}{Market capitalization data}{\autocite[6]{SurveysSoftware2015}}
The software industry possesses many unique idiosyncrasies stemming from the unusual nature of its products.
It is one of the largest and most dynamic industries in the United States, accounting for 19.5\% of the S\&P 500 in July 2015.\autocite[6]{SurveysSoftware2015}
The greatest difference between the software business and other, more traditional industries that produce tangible goods or services, is that software is trivially reproducible and production does not depend on limited inputs.\autocite[3]{buxmann2012software}
In most circumstances the only significant input cost to producing a software product is labour of the developers.
Even large corporations like Oracle are completely dependent upon engineers and software developers for building the company's software products and maintaining support contracts.\autocite[115]{finkle2012larry}
This also means that the barriers to entry into the market are low as any business with the perquisite human capital can produce services that compete directly with established providers.\autocite[115]{finkle2012larry}
Indeed, the high profit potential of the software business tends to attract many new entrants to the industry.\autocite[116]{finkle2012larry}
Futhermore, although the domestic market is the primary driver of revenue for most businesses,\autocite[]{ITSoftwareEconomist} the software market is also highly international in nature and software products are often developed by international teams.\autocite[3]{buxmann2012software}
%Add more to push to one page in length

\wrapfig{decisionMatrix}{r}{.41}{Decision matrix}{\Autocite[45]{buxmann2012software}}
This inherent reproducibility dominates almost every aspect of the industry, especially the dynamics of competition and the profit strategies of major software producers.
It is important to understand what guides individuals and business to choose to externally source their software.
In this regard the field of transaction theory is useful in understanding the decision making process of software consumers, see Figure \ref{fig:transactionTheory}.
In the broadest sense, transactions can be classified along a spectrum according to how specific they are to particular investments.
Markets are the most efficient tool to optimize transaction costs in situations where the specificity of the transactions is low.
On the other hand, more specific investments lead the transaction partners to grow more interdependent which in turn encourages opportunistic behavior as the parties have more leverage over each other.\autocite[44]{buxmann2012software}
Since the transaction costs involved in software redistribution are effectively negligible it almost always makes sense to externally source any software that is not highly specific to the particular needs of a given company.

\wrapfig{marketShareOfTypesOfSoftware}{r}{.5}{Market share of software types}{\autocite[7]{buxmann2012software}}
The shift towards internet-based services has had the effect of lowering transaction costs, therefore encouraging standardization of software from dedicated providers rather than homegrown solutions.\autocite[46]{buxmann2012software}\autocite[]{SurveysSoftware2015}
Cloud-based software in particular has seen dramatic and sustained growth in recent years, a trend which is expected to continue.
The projected compound annual growth rate of cloud software from 2013 to 2018 currently rests at 20\%, indicating a substantial shift within the industry. Significant players such as Adobe, Autodesk, Oracle, and Microsoft have realigned themselves to try take advantage of the ongoing shift towards cloud-based services.\autocite[40]{SurveysSoftware2015}
These shifts are only the latest manifestation of the industry's long term trend towards greater consolidation as improvements in technology benefit more sophisticated software providers positioned to take advantage of their negligible variable costs and decreasing transaction costs.\autocite[14]{buxmann2012software}

There are two primary ways for a software producer to profit from the distribution of their products: a business can sell software licenses, or it sell software and services by contract.\autocite[]{ValuelineOverview}
The business model of companies pursuing the latter strategy is usually to sell licenses which charge the customer proportional to some metric of usage.\autocite[14]{buxmann2012software}
Consulting fees are usually proportional to the number of working days hired or results while maintenance service fees are usually a flat fraction of the original licensing fee.\autocite[16]{buxmann2012software} See Figure \ref{fig:genericBusinessModels} and \ref{fig:parametersOfPricing}.

%There are two forces that tend to drive spending in the enterprise software industry, there are IT-driven initiatives and there are business driven initiatives.\autocite[]{NextGenBusinessSoftware}
Business driven IT developments tend to be outward facing and focus on concrete business problems rather than underlying technologies, and it is this sort of spending that drives much of current IT investment.\autocite[]{NextGenBusinessSoftware}
Enterprise resource management is another important sector of the software industry including business focused upon managing accounting, payroll, and human capital.\autocite[46]{SurveysSoftware2015}

